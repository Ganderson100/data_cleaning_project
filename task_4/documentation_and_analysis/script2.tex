\documentclass[]{article}
\usepackage{lmodern}
\usepackage{amssymb,amsmath}
\usepackage{ifxetex,ifluatex}
\usepackage{fixltx2e} % provides \textsubscript
\ifnum 0\ifxetex 1\fi\ifluatex 1\fi=0 % if pdftex
  \usepackage[T1]{fontenc}
  \usepackage[utf8]{inputenc}
\else % if luatex or xelatex
  \ifxetex
    \usepackage{mathspec}
  \else
    \usepackage{fontspec}
  \fi
  \defaultfontfeatures{Ligatures=TeX,Scale=MatchLowercase}
\fi
% use upquote if available, for straight quotes in verbatim environments
\IfFileExists{upquote.sty}{\usepackage{upquote}}{}
% use microtype if available
\IfFileExists{microtype.sty}{%
\usepackage{microtype}
\UseMicrotypeSet[protrusion]{basicmath} % disable protrusion for tt fonts
}{}
\usepackage[margin=1in]{geometry}
\usepackage{hyperref}
\hypersetup{unicode=true,
            pdftitle={Task 4 - Halloween Candy Data},
            pdfauthor={Greg Anderson, DE2 Data Analysis},
            pdfborder={0 0 0},
            breaklinks=true}
\urlstyle{same}  % don't use monospace font for urls
\usepackage{color}
\usepackage{fancyvrb}
\newcommand{\VerbBar}{|}
\newcommand{\VERB}{\Verb[commandchars=\\\{\}]}
\DefineVerbatimEnvironment{Highlighting}{Verbatim}{commandchars=\\\{\}}
% Add ',fontsize=\small' for more characters per line
\usepackage{framed}
\definecolor{shadecolor}{RGB}{248,248,248}
\newenvironment{Shaded}{\begin{snugshade}}{\end{snugshade}}
\newcommand{\AlertTok}[1]{\textcolor[rgb]{0.94,0.16,0.16}{#1}}
\newcommand{\AnnotationTok}[1]{\textcolor[rgb]{0.56,0.35,0.01}{\textbf{\textit{#1}}}}
\newcommand{\AttributeTok}[1]{\textcolor[rgb]{0.77,0.63,0.00}{#1}}
\newcommand{\BaseNTok}[1]{\textcolor[rgb]{0.00,0.00,0.81}{#1}}
\newcommand{\BuiltInTok}[1]{#1}
\newcommand{\CharTok}[1]{\textcolor[rgb]{0.31,0.60,0.02}{#1}}
\newcommand{\CommentTok}[1]{\textcolor[rgb]{0.56,0.35,0.01}{\textit{#1}}}
\newcommand{\CommentVarTok}[1]{\textcolor[rgb]{0.56,0.35,0.01}{\textbf{\textit{#1}}}}
\newcommand{\ConstantTok}[1]{\textcolor[rgb]{0.00,0.00,0.00}{#1}}
\newcommand{\ControlFlowTok}[1]{\textcolor[rgb]{0.13,0.29,0.53}{\textbf{#1}}}
\newcommand{\DataTypeTok}[1]{\textcolor[rgb]{0.13,0.29,0.53}{#1}}
\newcommand{\DecValTok}[1]{\textcolor[rgb]{0.00,0.00,0.81}{#1}}
\newcommand{\DocumentationTok}[1]{\textcolor[rgb]{0.56,0.35,0.01}{\textbf{\textit{#1}}}}
\newcommand{\ErrorTok}[1]{\textcolor[rgb]{0.64,0.00,0.00}{\textbf{#1}}}
\newcommand{\ExtensionTok}[1]{#1}
\newcommand{\FloatTok}[1]{\textcolor[rgb]{0.00,0.00,0.81}{#1}}
\newcommand{\FunctionTok}[1]{\textcolor[rgb]{0.00,0.00,0.00}{#1}}
\newcommand{\ImportTok}[1]{#1}
\newcommand{\InformationTok}[1]{\textcolor[rgb]{0.56,0.35,0.01}{\textbf{\textit{#1}}}}
\newcommand{\KeywordTok}[1]{\textcolor[rgb]{0.13,0.29,0.53}{\textbf{#1}}}
\newcommand{\NormalTok}[1]{#1}
\newcommand{\OperatorTok}[1]{\textcolor[rgb]{0.81,0.36,0.00}{\textbf{#1}}}
\newcommand{\OtherTok}[1]{\textcolor[rgb]{0.56,0.35,0.01}{#1}}
\newcommand{\PreprocessorTok}[1]{\textcolor[rgb]{0.56,0.35,0.01}{\textit{#1}}}
\newcommand{\RegionMarkerTok}[1]{#1}
\newcommand{\SpecialCharTok}[1]{\textcolor[rgb]{0.00,0.00,0.00}{#1}}
\newcommand{\SpecialStringTok}[1]{\textcolor[rgb]{0.31,0.60,0.02}{#1}}
\newcommand{\StringTok}[1]{\textcolor[rgb]{0.31,0.60,0.02}{#1}}
\newcommand{\VariableTok}[1]{\textcolor[rgb]{0.00,0.00,0.00}{#1}}
\newcommand{\VerbatimStringTok}[1]{\textcolor[rgb]{0.31,0.60,0.02}{#1}}
\newcommand{\WarningTok}[1]{\textcolor[rgb]{0.56,0.35,0.01}{\textbf{\textit{#1}}}}
\usepackage{graphicx,grffile}
\makeatletter
\def\maxwidth{\ifdim\Gin@nat@width>\linewidth\linewidth\else\Gin@nat@width\fi}
\def\maxheight{\ifdim\Gin@nat@height>\textheight\textheight\else\Gin@nat@height\fi}
\makeatother
% Scale images if necessary, so that they will not overflow the page
% margins by default, and it is still possible to overwrite the defaults
% using explicit options in \includegraphics[width, height, ...]{}
\setkeys{Gin}{width=\maxwidth,height=\maxheight,keepaspectratio}
\IfFileExists{parskip.sty}{%
\usepackage{parskip}
}{% else
\setlength{\parindent}{0pt}
\setlength{\parskip}{6pt plus 2pt minus 1pt}
}
\setlength{\emergencystretch}{3em}  % prevent overfull lines
\providecommand{\tightlist}{%
  \setlength{\itemsep}{0pt}\setlength{\parskip}{0pt}}
\setcounter{secnumdepth}{0}
% Redefines (sub)paragraphs to behave more like sections
\ifx\paragraph\undefined\else
\let\oldparagraph\paragraph
\renewcommand{\paragraph}[1]{\oldparagraph{#1}\mbox{}}
\fi
\ifx\subparagraph\undefined\else
\let\oldsubparagraph\subparagraph
\renewcommand{\subparagraph}[1]{\oldsubparagraph{#1}\mbox{}}
\fi

%%% Use protect on footnotes to avoid problems with footnotes in titles
\let\rmarkdownfootnote\footnote%
\def\footnote{\protect\rmarkdownfootnote}

%%% Change title format to be more compact
\usepackage{titling}

% Create subtitle command for use in maketitle
\providecommand{\subtitle}[1]{
  \posttitle{
    \begin{center}\large#1\end{center}
    }
}

\setlength{\droptitle}{-2em}

  \title{Task 4 - Halloween Candy Data}
    \pretitle{\vspace{\droptitle}\centering\huge}
  \posttitle{\par}
    \author{Greg Anderson, DE2 Data Analysis}
    \preauthor{\centering\large\emph}
  \postauthor{\par}
      \predate{\centering\large\emph}
  \postdate{\par}
    \date{14/11/2019}


\begin{document}
\maketitle

\hypertarget{introduction}{%
\section{Introduction}\label{introduction}}

This report includes:

\begin{itemize}
\tightlist
\item
  A brief introduction to the haloween candy dataset\\
\item
  A list of assumptions made\\
\item
  The steps taken to clean the data\\
\item
  The answers to the questions presented in the task brief\\
\item
  Other interesting analyses or conclusions
\end{itemize}

\includegraphics{images/candycorn.png}

\hypertarget{the-dataset}{%
\subsection{The dataset}\label{the-dataset}}

The halloween candy data is taken from a survey primarily used to
produce hierarchy data of the most popular candy.

The raw data was held in three .xlsx files covering survey responses
from 2015, 2016 and 2017.

The raw data and analysis can be accessed on the
\href{https://www.scq.ubc.ca/so-much-candy-data-seriously/}{Science
Creative Quarterly} website.

\hypertarget{assumptions}{%
\subsection{Assumptions}\label{assumptions}}

Project assumptions:

\begin{enumerate}
\def\labelenumi{\arabic{enumi}.}
\item
  The focus of the analysis was candy related questions from the
  survey.\\
\item
  Analysis of other survey questions was not required.\\
\item
  The data extracted and cleaned for analysis were:

  \begin{itemize}
  \tightlist
  \item
    age\\
  \item
    candy\\
  \item
    whether the respondent was trick or treating or not\\
  \item
    gender\\
  \item
    candy rating\\
  \item
    country\\
  \item
    year
  \end{itemize}
\end{enumerate}

\hypertarget{cleaning-the-data}{%
\subsection{Cleaning the data}\label{cleaning-the-data}}

A number of steps were taken to address significant data quality issues.

The steps included:

\begin{enumerate}
\def\labelenumi{\arabic{enumi}.}
\tightlist
\item
  An initial cleaning of the column names using the package
  \emph{`Janitor'} and the function `clean\_names()' to remove special
  characters and spaces and to convert to lower case.\\
\item
  Converting the data from wide to long format using the
  `pivot\_longer()' function from the `Tidyr' package. Specific column
  numbers were selected in each data set that contained candy names to
  be included in the pivot.\\
\item
  Dropped unwanted columns from the survey responses that were not
  required for analysis.\\
\item
  Renaming column names to age, candy\_name, rating,
  trick\_or\_treating, country, gender, \& year.\\
\item
  Added ``gender'' and ``country'' columns to the 2015 data with NAs
\item
  Added a ``year'' column to all three files in case further time series
  analysis was required.\\
\item
  Orderded the columns across the three files in preparation for
  binding.\\
\item
  Bound the rows across three files using `rbind()'.\\
\item
  The `age' column was changed from character type to integer to remove
  responses provided as a text string rather than a number e.g. ``old''.
  NAs replaced such instances.\\
\item
  For the `candy\_name' column in 2017 the following text was removed
  using str\_replace() - ``q6\_''
\item
  Hard coding changes to the ``country'' column to address the
  considerable variation in the way some countries were provided in
  survey responses. Also changed a significant number of responses to NA
  where responses were provided that were not recognised countries.\\
\item
  Identified outliers in the age column e.g 1800. Decided to replace any
  ages over 120 with NA.
\end{enumerate}

\hypertarget{analysis}{%
\section{Analysis}\label{analysis}}

\hypertarget{question-1}{%
\subsection{Question 1}\label{question-1}}

\textbf{What is the total number of candy ratings given across the three
years. (number of candy ratings, not number of raters. Don't count
missing values)}

\begin{Shaded}
\begin{Highlighting}[]
\CommentTok{# ordedred table of number of ratings }
\NormalTok{total_candy_ratings <-}\StringTok{ }\NormalTok{clean_candy }\OperatorTok
\StringTok{  }\KeywordTok{select}\NormalTok{(candy_name, rating) }\OperatorTok
\StringTok{  }\KeywordTok{drop_na}\NormalTok{() }\OperatorTok
\StringTok{  }\KeywordTok{summarise}\NormalTok{(}\DataTypeTok{number_of_ratings =} \KeywordTok{n}\NormalTok{()) }

\NormalTok{total_candy_ratings}
\end{Highlighting}
\end{Shaded}

\begin{verbatim}
## # A tibble: 1 x 1
##   number_of_ratings
##               <int>
## 1            727256
\end{verbatim}

\hypertarget{question-2}{%
\subsection{Question 2}\label{question-2}}

\textbf{What was the average age of people who are going out trick or
treating and the average age of people not going trick or treating?}

\begin{Shaded}
\begin{Highlighting}[]
\CommentTok{# average age trick or treating}
\NormalTok{average_age_trick_or_treating <-}\StringTok{ }\NormalTok{clean_candy }\OperatorTok
\StringTok{  }\KeywordTok{select}\NormalTok{(age, trick_or_treating) }\OperatorTok
\StringTok{  }\KeywordTok{drop_na}\NormalTok{() }\OperatorTok
\StringTok{  }\KeywordTok{filter}\NormalTok{(trick_or_treating }\OperatorTok{==}\StringTok{ "Yes"}\NormalTok{) }\OperatorTok
\StringTok{  }\KeywordTok{summarise}\NormalTok{(}\DataTypeTok{trick_or_treating_average_age =} \KeywordTok{mean}\NormalTok{(age))}

\NormalTok{average_age_trick_or_treating}
\end{Highlighting}
\end{Shaded}

\begin{verbatim}
## # A tibble: 1 x 1
##   trick_or_treating_average_age
##                           <dbl>
## 1                          35.2
\end{verbatim}

\begin{Shaded}
\begin{Highlighting}[]
\CommentTok{# average age not trick or treating}
\NormalTok{average_age_not_trick_or_treating <-}\StringTok{ }\NormalTok{clean_candy }\OperatorTok
\StringTok{  }\KeywordTok{select}\NormalTok{(age, trick_or_treating) }\OperatorTok
\StringTok{  }\KeywordTok{drop_na}\NormalTok{() }\OperatorTok
\StringTok{  }\KeywordTok{filter}\NormalTok{(trick_or_treating }\OperatorTok{==}\StringTok{ "No"}\NormalTok{) }\OperatorTok
\StringTok{  }\KeywordTok{summarise}\NormalTok{(}\DataTypeTok{not_trick_or_treating_average_age =} \KeywordTok{mean}\NormalTok{(age))}
  
\NormalTok{average_age_not_trick_or_treating}
\end{Highlighting}
\end{Shaded}

\begin{verbatim}
## # A tibble: 1 x 1
##   not_trick_or_treating_average_age
##                               <dbl>
## 1                              39.3
\end{verbatim}

\hypertarget{question-3}{%
\subsection{Question 3}\label{question-3}}

\textbf{For each of joy, despair and meh, which candy bar received the
most of these ratings?}

\begin{Shaded}
\begin{Highlighting}[]
\NormalTok{joy_ratings <-}\StringTok{ }\NormalTok{clean_candy }\OperatorTok
\StringTok{  }\KeywordTok{select}\NormalTok{(candy_name, rating) }\OperatorTok
\StringTok{  }\KeywordTok{drop_na}\NormalTok{() }\OperatorTok
\StringTok{  }\KeywordTok{filter}\NormalTok{(rating }\OperatorTok{==}\StringTok{ "JOY"}\NormalTok{) }\OperatorTok
\StringTok{  }\KeywordTok{group_by}\NormalTok{(candy_name) }\OperatorTok
\StringTok{  }\KeywordTok{summarise}\NormalTok{(}\DataTypeTok{number_of_joy_ratings =} \KeywordTok{max}\NormalTok{(}\KeywordTok{length}\NormalTok{(rating))) }\OperatorTok
\StringTok{  }\KeywordTok{top_n}\NormalTok{(}\DecValTok{1}\NormalTok{)}
\end{Highlighting}
\end{Shaded}

\begin{verbatim}
## Selecting by number_of_joy_ratings
\end{verbatim}

\begin{Shaded}
\begin{Highlighting}[]
\NormalTok{meh_ratings <-}\StringTok{ }\NormalTok{clean_candy }\OperatorTok
\StringTok{  }\KeywordTok{select}\NormalTok{(candy_name, rating) }\OperatorTok
\StringTok{  }\KeywordTok{drop_na}\NormalTok{() }\OperatorTok
\StringTok{  }\KeywordTok{filter}\NormalTok{(rating }\OperatorTok{==}\StringTok{ "MEH"}\NormalTok{) }\OperatorTok
\StringTok{  }\KeywordTok{group_by}\NormalTok{(candy_name) }\OperatorTok
\StringTok{  }\KeywordTok{summarise}\NormalTok{(}\DataTypeTok{number_of_meh_ratings =} \KeywordTok{max}\NormalTok{(}\KeywordTok{length}\NormalTok{(rating))) }\OperatorTok
\StringTok{  }\KeywordTok{top_n}\NormalTok{(}\DecValTok{1}\NormalTok{)}
\end{Highlighting}
\end{Shaded}

\begin{verbatim}
## Selecting by number_of_meh_ratings
\end{verbatim}

\begin{Shaded}
\begin{Highlighting}[]
\NormalTok{despair_ratings <-}\StringTok{ }\NormalTok{clean_candy }\OperatorTok
\StringTok{  }\KeywordTok{select}\NormalTok{(candy_name, rating) }\OperatorTok
\StringTok{  }\KeywordTok{drop_na}\NormalTok{() }\OperatorTok
\StringTok{  }\KeywordTok{filter}\NormalTok{(rating }\OperatorTok{==}\StringTok{ "DESPAIR"}\NormalTok{) }\OperatorTok
\StringTok{  }\KeywordTok{group_by}\NormalTok{(candy_name) }\OperatorTok
\StringTok{  }\KeywordTok{summarise}\NormalTok{(}\DataTypeTok{number_of_despair_ratings =} \KeywordTok{max}\NormalTok{(}\KeywordTok{length}\NormalTok{(rating))) }\OperatorTok
\StringTok{  }\KeywordTok{top_n}\NormalTok{(}\DecValTok{1}\NormalTok{)}
\end{Highlighting}
\end{Shaded}

\begin{verbatim}
## Selecting by number_of_despair_ratings
\end{verbatim}

\begin{Shaded}
\begin{Highlighting}[]
\NormalTok{joy_ratings}
\end{Highlighting}
\end{Shaded}

\begin{verbatim}
## # A tibble: 1 x 2
##   candy_name               number_of_joy_ratings
##   <chr>                                    <int>
## 1 any_full_sized_candy_bar                  7250
\end{verbatim}

\begin{Shaded}
\begin{Highlighting}[]
\NormalTok{meh_ratings}
\end{Highlighting}
\end{Shaded}

\begin{verbatim}
## # A tibble: 1 x 2
##   candy_name number_of_meh_ratings
##   <chr>                      <int>
## 1 lollipops                   1508
\end{verbatim}

\begin{Shaded}
\begin{Highlighting}[]
\NormalTok{despair_ratings}
\end{Highlighting}
\end{Shaded}

\begin{verbatim}
## # A tibble: 1 x 2
##   candy_name        number_of_despair_ratings
##   <chr>                                 <int>
## 1 broken_glow_stick                      7542
\end{verbatim}

\hypertarget{question-4}{%
\subsection{Question 4}\label{question-4}}

\textbf{How many people rated Starburst as despair?}

\begin{Shaded}
\begin{Highlighting}[]
\NormalTok{starburst_despair <-}\StringTok{ }\NormalTok{clean_candy }\OperatorTok
\StringTok{  }\KeywordTok{select}\NormalTok{(candy_name, rating) }\OperatorTok
\StringTok{  }\KeywordTok{filter}\NormalTok{(candy_name }\OperatorTok{==}\StringTok{ "starburst"}\NormalTok{) }\OperatorTok
\StringTok{  }\KeywordTok{filter}\NormalTok{(rating }\OperatorTok{==}\StringTok{ "DESPAIR"}\NormalTok{) }\OperatorTok
\StringTok{  }\KeywordTok{summarise}\NormalTok{(}\DataTypeTok{number_starburst_rated_despair =} \KeywordTok{n}\NormalTok{())}

\NormalTok{starburst_despair}
\end{Highlighting}
\end{Shaded}

\begin{verbatim}
## # A tibble: 1 x 1
##   number_starburst_rated_despair
##                            <int>
## 1                           1867
\end{verbatim}

\hypertarget{conclusions}{%
\section{Conclusions}\label{conclusions}}

Recommend improving data collection practices to at least prevent
respondents from entering free text.


\end{document}
